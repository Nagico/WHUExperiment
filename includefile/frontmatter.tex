% !Mode:: "TeX:UTF-8"

%%% 此部分需要自行填写: 中文摘要及关键词 

%%% 郑重声明部分无需改动

%%%---- 郑重声明 (无需改动)------------------------------------%
\newpage
\thispagestyle{empty}
\vspace*{20pt}
\begin{center}{\ziju{0.8}\pmb{\songti\zihao{2} 郑重声明}}\end{center}
\par\vspace*{30pt}
\renewcommand{\baselinestretch}{2}

{\zihao{4}%

本人呈交的设计报告,是在指导老师的指导下,独立进行实验工作所取得的成果,
所有数据、图片资料真实可靠。 尽我所知,除文中已经注明引用的内容外,
本设计报告不包含他人享有著作权的内容。
对本设计报告做出贡献的其他个人和集体,
均已在文中以明确的方式标明。本设计报告的知识产权归属于培养单位。\\[2cm]

\hspace*{1cm}本人签名: $\underline{\hspace{3.5cm}}$
\hspace{2cm}日期: $\underline{\hspace{3.5cm}}$\hfill\par}
%------------------------------------------------------------------------------
\baselineskip=23pt  % 正文行距为 23 磅
%------------------------------------------------------------------------------





%%======摘要===========================%
\begin{cnabstract}
\thispagestyle{empty}

HCI Lab: Assignment 3 的实验目的是自定义层次绘图功能,绘制相应的剪贴画(collage)。

实验设计主要遵循每个VisualElement包含的位置、大小以及parent-child信息,除此之外,该实验还遵循Canvas对象的绘图要求。

实验内容主要包括VisualElement.java的新增内容、BaseVisualElement.java的实现、基本绘制模块的实现、特殊布局的实现以及特殊的裁剪策略的实现。

实验结论为本次实验实现的功能能够较好地通过层次绘图的方式实现collage的绘制。


\end{cnabstract}
\par
\vspace*{2em}


%%%%--  关键词 -----------------------------------------%%%%%%%%
%%%%-- 注意: 每个关键词之间用“;”分开,最后一个关键词不打标点符号
\cnkeywords{层次绘图;布局;安卓;画布  }


%%====英文摘要==========================%

