\chapter{模板使用教程}
 
 \section{具体使用步骤}

 \begin{description}
  \item[Step 1]  打开主文档 main.tex,填写题目、学生姓名等等信息,书写正文。
  \item[Step 2]  进入 pages 文件夹,打开 frontmatter.tex,backmatter.tex 这两个文档,
  分别填写 (1) 中文摘要,(2) 实验结论。并根据实际情况创建chapter tex文件书写正文。
  \item[Step 3]  打开主文档,在正文部分修改include的文件,注意此时不需要tex后缀名。
  \item[Step 4]  导入bib参考文献,在文档中引用。 
  \item[Step 5]  使用 XeLaTeX 编译。
\end{description}

\section{文档类型选择}

{\textbf{\zihao{-2}{本小节是毕业论文打印介绍,实验报告可以略过}}}

文档类型有 2 种情形:

\begin{table}[ht]\centering
\begin{tabular}{ll}
\hline
   \verb|\documentclass{WHUBachelor}|                     &  毕业论文 \\
   \verb|\documentclass[forprint]{WHUBachelor}|        &  毕业论文打印版 \\
\hline
\end{tabular}
\end{table}
相关解释见下节.


\section{打印的问题}

{\textbf{\zihao{-2}{本小节是毕业论文打印介绍,实验报告可以略过}}}

\begin{enumerate}[i)] 
%  \item  论文要求\colorbox{yellow}{单面打印}.
  \item  关于文档选项 forprint:交付打印时,建议加上选项 forprint,以消除链接文字之彩色,避免打印字迹偏淡。
  \item  打印时留意不要缩小页面或居中。即页面放缩方式应该是“无”(Adobe Reader XI 是选择“实际大小”)。
           有可能页面放缩方式默认为“适合可打印区域”,会导致打印为原页面大小的 $97\%$。
           文字不要居中打印,是因为考虑到装订,左侧的空白留得稍多一点(模板已作预留)。
  \item  遗留问题:封面需要打印部重新制作。校内打印部通常有现成的模板。
\end{enumerate}
%如果不是彩色打印机,请在打印时,选择将彩色打印为黑白,否则彩色文字打出的墨迹会偏淡.

\textbf{问}: {\kaishu 生成 PDF 文件时,不能去掉目录和文章的引用彩色方框,请问怎么解决?}

\textbf{答}: {\kaishu 方框表示超级链接,只在电脑上看得见. 实际打印时,是没有的. 另外,文档类型加选项 forprint 之后,这些框框会隐掉的。}

 \vfill

本文档下载更新地址:\url{https://github.com/Nagico/WHUExperiment}. 使用之前,请移步查看是否有更新。