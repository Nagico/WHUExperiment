\chapter{先说重要的}
 
 \section{具体使用步骤}

 \begin{description}

  \item[Step 1]  进入 includefile 文件夹,  打开 frontmatter.tex, backmatter.tex 这两个文档,
        分别填写 (1) 中文摘要, (2) 实验结论.

  \item[Step 2]  打开主文档 Experiment-template.tex, 填写题目、学生姓名等等信息, 书写正文.

  \item[Step 3]  使用 XeLaTeX 编译. 具体见 \ref{sec-compile} 节.


\end{description}



\section{编译的方法}\label{sec-compile}

默认使用 XeLaTeX 编译, 直接生成~pdf 文件.

若另存为新文档, 请确保文档保存类型为 \verb|:UTF-8|. 当然目前很多编辑器默认文字编码为 UTF-8.
WinEdt 9.0 之后的版本都是默认保存为 UTF-8 的.


%使用~XeLaTeX 编译, 直接生成~pdf 文件.
%pdf 文件也可以反向搜索! \CJKunderwave{双击~pdf 中要修改的文字, 将直接跳转到源文件中相应位置.}
%





\section{文档类型选择}

{\textbf{\zihao{-2}{本小节是毕业论文打印介绍,实验报告可以略过}}}

文档类型有 2 种情形:

\begin{table}[ht]\centering
\begin{tabular}{ll}
\hline
   \verb|\documentclass{WHUBachelor}|                     &  毕业论文 \\
   \verb|\documentclass[forprint]{WHUBachelor}|        &  毕业论文打印版 \\
\hline
\end{tabular}
\end{table}
相关解释见下节.


\section{打印的问题}

{\textbf{\zihao{-2}{本小节是毕业论文打印介绍,实验报告可以略过}}}

\begin{enumerate}[i)] 
%  \item  论文要求\colorbox{yellow}{单面打印}.
  \item  关于文档选项 forprint: 交付打印时, 建议加上选项 forprint, 以消除链接文字之彩色, 避免打印字迹偏淡.
  \item  打印时留意不要缩小页面或居中. 即页面放缩方式应该是``无''(Adobe Reader XI 是选择``实际大小'').
           有可能页面放缩方式默认为``适合可打印区域'', 会导致打印为原页面大小的 $97\%$.
           文字不要居中打印, 是因为考虑到装订, 左侧的空白留得稍多一点(模板已作预留).
  \item  遗留问题: 封面需要打印部重新制作.  校内打印部通常有现成的模板.
           我们自己做的封面, 打印部不一定好用.
\end{enumerate}
%如果不是彩色打印机, 请在打印时, 选择将彩色打印为黑白, 否则彩色文字打出的墨迹会偏淡.

\textbf{问}: {\kaishu 生成 PDF 文件时,不能去掉目录和文章的引用彩色方框,请问怎么解决?}

\textbf{答}: {\kaishu 方框表示超级链接, 只在电脑上看得见. 实际打印时, 是没有的. 另外, 文档类型加选项 forprint 之后, 这些框框会隐掉的. }

 \vfill

本文档下载更新地址: \url{https://github.com/xiaoxinganling/WHUExperiment}. 使用之前, 请移步查看是否有更新.

问题反馈及建议, 请联系: mxzhou1998@gmail.com.