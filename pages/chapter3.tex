\chapter{模板使用教程}
 
\section{Readme}

模板文件的结构, 如下表所示:
 \begin{table}[ht]\centering
\begin{tabular}{r|r|l}
	\hline\hline
	\multicolumn{2}{l|}{main.tex }       & 主文档. 在其中填写首页信息、正文引用、参考文献引用。             \\ \hline
                                    & chapter$x$.tex & 第$x$章节正文。             \\ \cline{2-3}
    \raisebox{1em}{pages 文件夹}   & frontmatter.tex & 郑重声明、摘要。               \\ \cline{2-3}
	 &  backmatter.tex & 实验结论.                       \\ \hline
	\multicolumn{2}{l|}{figures 文件夹}                  & 存放图片文件。                   \\ \hline
    \multicolumn{2}{l|}{ref 文件夹}                  & 存放Bib参考文献文件。                   \\ \hline
	\multicolumn{2}{l|}{WHUBachelor.cls }             & 定义文档格式的 class file。不可删除。 \\ \hline\hline
\end{tabular}
\end{table}

无需也不要改变、移动上述文档的位置。

如果不习惯用~\verb|\include{ }|~的方式加入“子文档”,当然可以把它们合并在主文档,成为一个文档。
({\kaishu 但是这样并不会给我们带来方便。})

 \section{具体使用步骤}

 \begin{description}
  \item[Step 1]  打开主文档 main.tex,填写题目、学生姓名等等信息,书写正文。
  \item[Step 2]  进入 pages 文件夹,打开 frontmatter.tex,backmatter.tex 这两个文档,
  分别填写 (1) 中文摘要,(2) 实验结论。并根据实际情况创建chapter tex文件书写正文。
  \item[Step 3]  打开主文档,在正文部分修改include的文件,注意此时不需要tex后缀名。
  \item[Step 4]  导入bib参考文献,在文档中引用。 
  \item[Step 5]  使用 XeLaTeX 编译。
\end{description}

\section{其他}

自此,~\LaTeX~安装与模板使用说明已介绍完毕。接下来的章节会以本模板支持的命令为例,讲解~\LaTeX~的使用方法。请将PDF配合Tex源码一同学习,可自行修改相关命令,查看效果。

其中 Chapter \ref{cha:latex-brief-intro}将简要的介绍~\LaTeX~使用方法,包括多级标题、字体样式、字号调节、定理与公式、图片与表格的简单使用。初次学习可以参考此章节。


 \vfill

本文档下载更新地址:\url{https://github.com/Nagico/WHUExperiment}. 使用之前,请移步查看是否有更新。