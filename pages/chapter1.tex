\chapter{~\LaTeX~ 介绍}

~\LaTeX~是一种基于Tex的排版系统,它不像Word软件编写文件一样所见即所得,而是用一定的语法或者标记符号来组织内容。~\LaTeX~在学术写作中被广泛使用,特别是像数学和计算机这样的学科。~\LaTeX~可以让你忘记格式,而专注于内容。

有人可能会问我们已经有Word了,用起来也很方便啊,为什么还要用~\LaTeX~这种还有些技术门槛的工具呢?其实在学术写作中,我们往往会对内容不停地改来改去,特别是如果还插入了图片的话,每次修改都可能需要重新排版。而~\LaTeX~可以让你不用担心这些,任何时候都能帮你输出高质量的排版。

\section{~\LaTeX~优点}

经常有人喜欢对比 ~\LaTeX~ 和以 Word 为代表的“所见即所得”(What You See Is What You Get)字处理工具。这种对比是没有意义的,因为 TEX 是一个排版引擎,~\LaTeX~ 是其封装,而 Word 是字处理工具。二者的设计目标不一致,也各自有自己的适用范围。 不过,这里仍旧总结 ~\LaTeX~ 的一些优点:

%\cite{axiangBaYiBaLaTeXHeWordXiangBiQiYouQueDian2021}

\begin{itemize}
  \item 具有专业的排版输出能力
  \item 具有方便而强大的数学公式排版能力
  \item 绝大多数时候,用户只需专注于一些组织文档结构的基础命令,无需(或很少)操心文档 的版面设计
  \item 很容易生成复杂的专业排版元素,如脚注、交叉引用、参考文献、目录等
  \item 强大的可扩展性,世界各地的人开发了数以千计的 ~\LaTeX~ 宏包用于补充和扩展 ~\LaTeX~ 的功能
  \item 能够促使用户写出结构良好的文档,而这也是 ~\LaTeX~ 存在的初衷
\end{itemize}

\section{~\LaTeX~缺点}

同时不可否认的是,~\LaTeX~的使用需要一定的学习门槛,同时在使用过程中存在以下缺点:

\begin{itemize}
  \item 不容易排查错误。~\LaTeX~ 作为一个依靠编写代码工作的排版工具,其使用的宏语言比 C++ 或 Python 等程序设计语言在错误排查方面困难得多。它虽然能够提示错误,但不提供调试的机制,有时错误提示还很难理解。
  \item 不容易定制样式。~\LaTeX~ 提供了一个基本上良好的样式,为了让用户不去关注样式而专注于文档结构。但如果想要改进 ~\LaTeX~ 生成的文档样式则是十分困难,需要系统的学习~\LaTeX~排版。
  \item 相比“所见即所得”的模式有一些不便,为了查看生成文档的效果,用户总要不停地编译。
\end{itemize}

