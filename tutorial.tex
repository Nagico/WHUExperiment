% !Mode:: "TeX:UTF-8"
%% 请使用 XeLaTeX 编译本文.
% \documentclass{WHUBachelor}% 选项 forprint: 交付打印时添加, 避免彩色链接字迹打印偏淡. 即使用下一行:
\documentclass[forprint]{WHUBachelor}
%---------------------这里添加所需的package--------------------------------
\usepackage{url}

%--------------------------------------------------------------------------
\makeatletter
\def\BState{\State\hskip-\ALG@thistlm}
\makeatother
\begin{document}
%-----------------------------------------------------------------------------

%%%%%%% 下面的内容, 据实填空.

\Ccoursename{XXX课程} %课程名称
\title{ ~\LaTeX~ 模板及使用教程\\Introduction of ~\LaTeX~ Template} %实验名称 换行请使用\\
\author{} % 学生姓名
\Csupervisor{XXXX \quad 教授} %指导教师一姓名、职称

% 默认不显示指导教师二,需要时可在WHUBachelor.cls 80+行处将"关闭第二指导教师显示"下一行的注释解除
\CsupervisorAnother{无} %指导教师二姓名、职称 

\CstudentNum{XXXX} %学号
\Cmajor{XXXX} % 专业名称
\date{二〇一九年六月} % 日期

%-----------------------------------------------------------------------------

\pdfbookmark[0]{封面}{title}         % 封面页加到 pdf 书签
\maketitle
\frontmatter
\pagenumbering{Roman}              % 正文之前的页码用大写罗马字母编号.2019.6.16:更新 正文之前的页码隐藏,无需显示
%-----------------------------------------------------------------------------
% !Mode:: "TeX:UTF-8"

%%% 此部分需要自行填写: 中文摘要及关键词 

%%% 郑重声明部分无需改动

%%%---- 郑重声明 (无需改动)------------------------------------%
\newpage
\thispagestyle{empty}
\vspace*{20pt}
\begin{center}{\ziju{0.8}\pmb{\songti\zihao{2} 郑重声明}}\end{center}
\par\vspace*{30pt}
\renewcommand{\baselinestretch}{2}

{\zihao{4}%

本人呈交的设计报告,是在指导老师的指导下,独立进行实验工作所取得的成果,
所有数据、图片资料真实可靠。 尽我所知,除文中已经注明引用的内容外,
本设计报告不包含他人享有著作权的内容。
对本设计报告做出贡献的其他个人和集体,
均已在文中以明确的方式标明。本设计报告的知识产权归属于培养单位。\\[2cm]

\hspace*{1cm}本人签名: $\underline{\hspace{3.5cm}}$
\hspace{2cm}日期: $\underline{\hspace{3.5cm}}$\hfill\par}
%------------------------------------------------------------------------------
\baselineskip=23pt  % 正文行距为 23 磅
%------------------------------------------------------------------------------





%%======摘要===========================%
\begin{cnabstract}
\thispagestyle{empty}

本文使用武汉大学计算机学院实验报告的~\LaTeX~模板,并介绍~\LaTeX~和模板的使用。

\begin{itemize}
  \item 本项目仓库地址:\url{https://github.com/Nagico/WHUExperiment}
  \item 参考repo:
  \begin{enumerate}
    \item \url{https://github.com/whutug/whu-thesis}
    \item \url{https://github.com/xiaoxinganling/WHUExperiment}
  \end{enumerate}
\end{itemize}

欢迎进入仓库中给开发者一个免费的star\textasciitilde


\end{cnabstract}
\par
\vspace*{2em}


%%%%--  关键词 -----------------------------------------%%%%%%%%
%%%%-- 注意: 每个关键词之间用“;”分开,最后一个关键词不打标点符号
\cnkeywords{实验报告; \LaTeX{}; 模板   }



    % 加入摘要, 申明.
%==========================把目录加入到书签==============================%%%%%%


\tableofcontents
\thispagestyle{empty}				%不显示罗马数字 ——zmx更新于2019.06.18
\addtocontents{toc}{\protect\thispagestyle{empty}}




\mainmatter %% 以下是正文
%%%%%%%%%%%%%%%%%%%%%%%%%%%--------main matter-------%%%%%%%%%%%%%%%%%%%%%%%%%%%%%%%%%%%%
\pagestyle{plain}%plain
%\cfoot{\thepage{\zihao{5}\bf\usefonttimes}}
%\renewcommand{\baselinestretch}{1.6}
%\setlength{\baselineskip}{23pt}
\baselineskip=23pt  % 正文行距为 23 磅

%此处书写正文-------------------------------------------------------------------------------------

\chapter{~\LaTeX~ 介绍}

~\LaTeX~是一种基于Tex的排版系统,它不像Word软件编写文件一样所见即所得,而是用一定的语法或者标记符号来组织内容。~\LaTeX~在学术写作中被广泛使用,特别是像数学和计算机这样的学科。~\LaTeX~可以让你忘记格式,而专注于内容。

有人可能会问我们已经有Word了,用起来也很方便啊,为什么还要用~\LaTeX~这种还有些技术门槛的工具呢?其实在学术写作中,我们往往会对内容不停地改来改去,特别是如果还插入了图片的话,每次修改都可能需要重新排版。而~\LaTeX~可以让你不用担心这些,任何时候都能帮你输出高质量的排版。

\section{~\LaTeX~优点}

经常有人喜欢对比 ~\LaTeX~ 和以 Word 为代表的“所见即所得”(What You See Is What You Get)字处理工具。这种对比是没有意义的,因为 TEX 是一个排版引擎,~\LaTeX~ 是其封装,而 Word 是字处理工具。二者的设计目标不一致,也各自有自己的适用范围。 不过,这里仍旧总结 ~\LaTeX~ 的一些优点:

%\cite{axiangBaYiBaLaTeXHeWordXiangBiQiYouQueDian2021}

\begin{itemize}
    \item 具有专业的排版输出能力
    \item 具有方便而强大的数学公式排版能力
    \item 绝大多数时候,用户只需专注于一些组织文档结构的基础命令,无需(或很少)操心文档 的版面设计
    \item 很容易生成复杂的专业排版元素,如脚注、交叉引用、参考文献、目录等
    \item 强大的可扩展性,世界各地的人开发了数以千计的 ~\LaTeX~ 宏包用于补充和扩展 ~\LaTeX~ 的功能
    \item 能够促使用户写出结构良好的文档,而这也是 ~\LaTeX~ 存在的初衷
\end{itemize}

\section{~\LaTeX~缺点}

同时不可否认的是,~\LaTeX~的使用需要一定的学习门槛,同时在使用过程中存在以下缺点:

\begin{itemize}
    \item 不容易排查错误。~\LaTeX~ 作为一个依靠编写代码工作的排版工具,其使用的宏语言比 C++ 或 Python 等程序设计语言在错误排查方面困难得多。它虽然能够提示错误,但不提供调试的机制,有时错误提示还很难理解。
    \item 不容易定制样式。~\LaTeX~ 提供了一个基本上良好的样式,为了让用户不去关注样式而专注于文档结构。但如果想要改进 ~\LaTeX~ 生成的文档样式则是十分困难,需要系统的学习~\LaTeX~排版。
    \item 相比“所见即所得”的模式有一些不便,为了查看生成文档的效果,用户总要不停地编译。
\end{itemize}


\chapter{~\LaTeX~的安装}

~\LaTeX~的使用需要安装相关的软件,目前主要使用的有两种方式进行~\LaTeX~编辑:
\begin{itemize}
  \item 在线编辑器(Overleaf、TeXPage)
  \item 本地编辑器(VSCode+插件、TeXShop)
\end{itemize}

个人推荐使用Overleaf进行编辑,无需安装本地编译环境,同时还可以进行多人协同操作。但使用在线编辑器的缺点就是必须连接网络。

\section{Overleaf的使用}

进入到Overleaf首页:\url{https://www.overleaf.com},点击右上角Register注册新账户。登录成功后如图\ref{fig:2-overlead-home}所示,会进入到项目界面。

\begin{figure}[htb]
  \centering
  \includegraphics[width=0.9\textwidth]{figures/chapter2/overleaf-home.png}
  \caption{Overleaf项目页面}
  \label{fig:2-overlead-home}
\end{figure}

此时你需要将使用的模板下载至本地。以此项目为例,进入\url{https://github.com/Nagico/WHUExperiment},点击Download ZIP即可将模板下载到本地。该模板同时也一同放至本文档旁,可以直接使用,但仍建议从Github上下载最新版本的模板。

\begin{figure}[htb]
    \centering
    \includegraphics[width=0.95\textwidth]{figures/chapter2/download-repo.png}
    \caption{下载模板}
    \label{fig:2-github-download}
\end{figure}

在Overleaf页面点击左侧的New Project,选择Upload Project,将下载的ZIP文件上传,即可将模板导入至Overleaf。

\begin{figure}[htb]
  \centering
  \includegraphics[width=0.3\textwidth]{figures/chapter2/upload-project.png}
  \caption{导入模板}
  \label{fig:2-upload-project}
\end{figure}

导入后会自动跳转到编辑界面,需要点击左上角的Menu进入设置界面,将Compiler修改为XeLatex以支持中文(图\ref{fig:2-compiler})。

\begin{figure}[H]
  \centering
  \begin{subfigure}{0.55\textwidth}
    \includegraphics[width=\linewidth]{figures/chapter2/menu.png}
    \caption{Menu按钮}
    \label{fig:2-menu}
  \end{subfigure}\qquad
  \begin{subfigure}{0.3\textwidth}
    \includegraphics[width=\linewidth]{figures/chapter2/xelatex.png}
    \caption{选择Compiler}
    \label{fig:2-compiler}
  \end{subfigure}
  \caption{配置~\LaTeX~}
  \label{fig:2-latex-conf}
\end{figure}

修改成功后点击Recompile重新编辑,即可正常使用。你可以在左侧进行项目文件的管理,后续所需的图片可以在相应文件夹处右键,选择Upload上传。

\begin{figure}[htb]
  \centering
  \includegraphics[width=0.95\textwidth]{figures/chapter2/overleaf-edit.png}
  \caption{Overleaf编辑页面}
  \label{fig:2-overleaf-edit}
\end{figure}

\section{本地编辑器}

\subsection{~\LaTeX~环境安装}

在清华源中下载Tex Live镜像文件:\url{https://mirrors.tuna.tsinghua.edu.cn/CTAN/systems/texlive/Images/texlive.iso},下载成功后双击挂载iso文件。


\begin{figure}[htb]
  \centering
  \includegraphics[width=0.95\textwidth]{figures/chapter2/texlive-iso.png}
  \caption{Tex Live镜像}
  \label{fig:2-texlive-iso}
\end{figure}

Windows下直接打开install-tl-windows.bat,Linux/Mac用户在终端下输入:

\begin{lstlisting}[language=bash]
  ./install-tl
\end{lstlisting}

使用默认配置进行安装。

\begin{figure}[htb]
  \centering
  \includegraphics[width=0.4\textwidth]{figures/chapter2/texlive-install.png}
  \caption{Tex Live安装}
  \label{fig:2-texlive-install}
\end{figure}

详细安装过程和注意事项请参考:\url{https://github.com/OsbertWang/install-latex-guide-zh-cn/releases/latest/}

\subsection{VSCode配置}

Tex Live自带的编辑器不太好用,个人一般使用VSCode配合LaTeX Workshop插件。

\begin{enumerate}
  \item 点击拓展图标,打开拓展
  \item 输入"latex workshop",选择第一个LaTeX Workshop插件
  \item 点击"install"进行安装,等待安装完成(如图\ref{fig:2-plugin-install})
\end{enumerate}

\begin{figure}[H]
  \centering
  \includegraphics[width=0.95\textwidth]{figures/chapter2/plugin-install.png}
  \caption{LaTeX Workshop插件安装}
  \label{fig:2-plugin-install}
\end{figure}

\begin{enumerate}
  \item 点击设置图标
  \item 点击设置
  \item 转到 UI 设置页面(如图\ref{fig:2-vscode-settings})
  \item 点击右上侧图标打开JSON配置文件,进入代码设置页面(如图\ref{fig:2-vscode-json})
\end{enumerate}

\begin{figure}[H]
  \centering
  \begin{subfigure}{0.45\textwidth}
    \includegraphics[width=\linewidth]{figures/chapter2/vscode-settings.png}
    \caption{UI设置界面}
    \label{fig:2-vscode-settings}
  \end{subfigure}\qquad
  \begin{subfigure}{0.45\textwidth}
    \includegraphics[width=\linewidth]{figures/chapter2/vscode-json.png}
    \caption{JSON配置文件}
    \label{fig:2-vscode-json}
  \end{subfigure}
  \caption{配置~\LaTeX~}
  \label{fig:2-vscode-conf}
\end{figure}

在JSON文件内输入以下内容:

由于PDF内代码不好复制,JSON内容请参考:\url{https://zhuanlan.zhihu.com/p/166523064}\quad \textbf{6.1 LaTeX配置代码展示}处

\begin{lstlisting}[language=json]
  {
    "latex-workshop.latex.autoBuild.run": "never",
    "latex-workshop.showContextMenu": true,
    "latex-workshop.intellisense.package.enabled": true,
    "latex-workshop.message.error.show": false,
    "latex-workshop.message.warning.show": false,
    "latex-workshop.latex.tools": [
        {
            "name": "xelatex",
            "command": "xelatex",
            "args": [
                "-synctex=1",
                "-interaction=nonstopmode",
                "-file-line-error",
                "%DOCFILE%"
            ]
        },
        {
            "name": "pdflatex",
            "command": "pdflatex",
            "args": [
                "-synctex=1",
                "-interaction=nonstopmode",
                "-file-line-error",
                "%DOCFILE%"
            ]
        },
        {
            "name": "latexmk",
            "command": "latexmk",
            "args": [
                "-synctex=1",
                "-interaction=nonstopmode",
                "-file-line-error",
                "-pdf",
                "-outdir=%OUTDIR%",
                "%DOCFILE%"
            ]
        },
        {
            "name": "bibtex",
            "command": "bibtex",
            "args": [
                "%DOCFILE%"
            ]
        }
    ],
    "latex-workshop.latex.recipes": [
        {
            "name": "XeLaTeX",
            "tools": [
                "xelatex"
            ]
        },
        {
            "name": "PDFLaTeX",
            "tools": [
                "pdflatex"
            ]
        },
        {
            "name": "BibTeX",
            "tools": [
                "bibtex"
            ]
        },
        {
            "name": "LaTeXmk",
            "tools": [
                "latexmk"
            ]
        },
        {
            "name": "xelatex -> bibtex -> xelatex*2",
            "tools": [
                "xelatex",
                "bibtex",
                "xelatex",
                "xelatex"
            ]
        },
        {
            "name": "pdflatex -> bibtex -> pdflatex*2",
            "tools": [
                "pdflatex",
                "bibtex",
                "pdflatex",
                "pdflatex"
            ]
        },
    ],
    "latex-workshop.latex.clean.fileTypes": [
        "*.aux",
        "*.bbl",
        "*.blg",
        "*.idx",
        "*.ind",
        "*.lof",
        "*.lot",
        "*.out",
        "*.toc",
        "*.acn",
        "*.acr",
        "*.alg",
        "*.glg",
        "*.glo",
        "*.gls",
        "*.ist",
        "*.fls",
        "*.log",
        "*.fdb_latexmk"
    ],
    "latex-workshop.latex.autoClean.run": "onFailed",
    "latex-workshop.latex.recipe.default": "lastUsed",
    "latex-workshop.view.pdf.internal.synctex.keybinding": "double-click"
}
\end{lstlisting}

\subsection{VSCode编译}

此时你需要将使用的模板下载至本地。以此项目为例,进入\url{https://github.com/Nagico/WHUExperiment},点击Download ZIP即可将模板下载到本地。该模板同时也一同放至本文档旁,可以直接使用,但仍建议从Github上下载最新版本的模板。

\begin{figure}[htb]
  \centering
  \includegraphics[width=0.95\textwidth]{figures/chapter2/download-repo.png}
  \caption{下载模板}
  \label{fig:2-github-download-2}
\end{figure}

将项目解压后用VSCode打开文件夹,点击选中 tex 文件,进行文件内容查看。

\begin{figure}[H]
  \centering
  \includegraphics[width=0.9\textwidth]{figures/chapter2/vscode-open.png}
  \caption{打开项目文件夹}
  \label{fig:2-vscode-open-folder}
\end{figure}

\begin{figure}[H]
  \centering
  \includegraphics[width=0.9\textwidth]{figures/chapter2/vscode-openfile.png}
  \caption{打开Tex文件}
  \label{fig:2-vscode-open-file}
\end{figure}

由于项目中会涉及参考文献的引用(.bib的编译),故而选择xelatex -> bibtex -> xelatex*2编译链。

\begin{figure}[htb]
  \centering
  \includegraphics[width=0.8\textwidth]{figures/chapter2/vscode-compile.png}
  \caption{Tex编译}
  \label{fig:2-vscode-tex-compile}
\end{figure}

点击编辑界面的右上角图标,即可查看编译结果。

\begin{figure}[H]
  \centering
  \includegraphics[width=0.95\textwidth]{figures/chapter2/vscode-edit.png}
  \caption{Tex编译结果}
  \label{fig:2-vscode-edit}
\end{figure}

更多VSCode配置可参考网上Blog,实现双向自动跳转等功能。新版VSCode、插件与以前有一定差别,最好选择2022年及以后较新的Blog学习。
\chapter{模板使用教程}
 
\section{Readme}

模板文件的结构, 如下表所示:
 \begin{table}[ht]\centering
\begin{tabular}{r|r|l}
	\hline\hline
	\multicolumn{2}{l|}{main.tex }       & 主文档. 在其中填写首页信息、正文引用、参考文献引用             \\ \hline
                                    & chapter$x$.tex & 第$x$章节正文            \\ \cline{2-3}
                                    & appendix$x$.tex & 附录$x$             \\ \cline{2-3}
    \raisebox{1em}{pages 文件夹}   & frontmatter.tex & 郑重声明、摘要               \\ \cline{2-3}
	 &  backmatter.tex & 实验结论                      \\ \hline
	\multicolumn{2}{l|}{figures 文件夹}                  & 存放图片文件               \\ \hline
    \multicolumn{2}{l|}{ref 文件夹}                  & 存放bib参考文献文件                   \\ \hline
	\multicolumn{2}{l|}{WHUExperiment.cls }             & 定义文档格式的 class file, 不可删除 \\ \hline\hline
\end{tabular}
\end{table}

无需也不要改变、移动上述文档的位置。

如果不习惯用~\verb|\include{ }|~的方式加入“子文档”,当然可以把它们合并在主文档,成为一个文档。
({\kaishu 但是这样并不会给我们带来方便。})

 \section{具体使用步骤}

 \begin{description}
  \item[Step 1]  打开主文档 main.tex,填写题目、学生姓名等等信息,书写正文。
  \item[Step 2]  进入 pages 文件夹,打开 frontmatter.tex,backmatter.tex 这两个文档,
  分别填写 (1) 中文摘要,(2) 实验结论。并根据实际情况创建chapter tex文件书写正文。
  \item[Step 3]  打开主文档,在正文部分修改include的文件,注意此时不需要tex后缀名。
  \item[Step 4]  导入bib参考文献,在文档中引用。 
  \item[Step 5]  使用 XeLaTeX 编译。
\end{description}

\section{其他}

自此,~\LaTeX~安装与模板使用说明已介绍完毕。接下来的章节会以本模板支持的命令为例,讲解~\LaTeX~的使用方法。请将PDF配合Tex源码一同学习,可自行修改相关命令,查看效果。

其中 Chapter \ref{cha:latex-brief-intro}将简要的介绍~\LaTeX~使用方法,包括多级标题、字体样式、字号调节、定理与公式、图片与表格的简单使用。初次学习可以参考此章节。


 \vfill

本文档下载更新地址:\url{https://github.com/Nagico/WHUExperiment}. 使用之前,请移步查看是否有更新。
\chapter{简要~\LaTeX~使用说明}\label{cha:latex-brief-intro}

\section{控制单位}

LaTeX里可使用的单位包括:
\begin{itemize}
    \item 毫米\ \verb|1mm|
    \item 厘米\ \verb|1cm|
    \item 英寸\ \verb|1in|
    \item 像素\ \verb|1pt|
    \item 基础文本尺寸的宽度\ \verb|1em|
    \item 基础文本尺寸的高度\ \verb|1ex|
    \item 字符宽度或行宽度的百分比值\ \verb|0.2\textwidth|
    \item 字符高度或行高度的百分比值\ \verb|0.2\textheight|
\end{itemize}

\section{特殊符号}

~\LaTeX~中有些符号无法正常显示,比如:\# \quad \% \quad \& \quad \{ \quad \} \quad \~{} \quad \_{} \quad \textbackslash \quad \backslash \quad \^{} \quad \$ 等控制符号,错误的在代码中使用会产生Error报错,并且很难发现错误原因。同时输入“\backslash 命令”时需注意,若命令结束后紧跟正文内容,需要在结尾加入空格,即\verb|a \quad b|。

发现Error报错后,请对刚才添加的内容进行错误排查,重点观察报错位置附近的特殊符号、插入图片的路径、代码 \} 大括号是否完整。若无法判断错误位置,可以将添加部分临时删除,一点一点的加入并编译,直至出现Error报错,缩小错误代码的位置。

\subsection{空白符号}

\paragraph*{~\LaTeX~空白的原则:}

\begin{itemize}
    \item 空行分段,多个空行等同为1个
    \item 段落首行缩进是自动的,绝对不能使用空格代替
    \item 英文中多个空格处理为1个空格,中文中空格将被忽略
    \item 汉字与其他字符的间距会自动有XeLaTeX处理
    \item 禁止使用中文全角空格
\end{itemize}

\paragraph*{空格的输入方法:}

\begin{itemize}
    \item 当前字体的一个宽度,即1em\\
    \verb|a \quad b| \quad a \quad b
    \item 2em空格\\
    \verb|a \qquad b| \quad a \qquad b
    \item 1/6em空格\\
    \verb|a \, b| \qquad \qquad  a\,b \\
    \verb|a \thinspace b| a \thinspace b
    \item 0.5em空格\\
    \verb|a \enspace b| \quad a \enspace b
    \item 空格\\
    \verb|a \ b| \quad a \ b
    \item 硬空格(不能分割\\
    \verb|a~b| \quad a~b
    \item 指定宽度的空白 (1pc=12pt=4.218mm)\\
    \verb|a \kern 1pc b| \ \qquad a\kern 1pc b\\
    \verb|a \kern -1em b| \qquad a\kern -1em b\\
    \verb|a \hskip 1em b| \qquad a\hskip 1em b\\
    \verb|a \hspace{35pt}b| \quad a\hspace{35pt}b
    \item 占位宽度 xyz的占位宽度\\
    \verb|a \hphantom{xyz}b| \quad a\hphantom{xyz}b
    \item 弹性宽度,占满横向空间\\
    \verb|a \hfill b| \quad a\hfill b
\end{itemize}
 
\subsection{文本控制}

\begin{itemize}
    \item 分段(有段首缩进) \\
    \verb|para1|\\ \hspace*{\fill} \\\verb|para2|

    para1

    para2

    \item 文字之间直接换行(无缩进)\\
    \verb|para1\\line2|

    para1\\line2

    \item 添加空白行(会出现Warning) \\
    \verb|line1 \\ ~\\ line2| \\
    line1 \\ ~\\ line2

    \item 这个命令也可以加空行(无Warning),一般用在文字之间加空行 \\
    \verb|line1 \\ \hspace*{\fill} \\ line2| \\
    line1\\ \hspace*{\fill} \\line2

    \item 控制空行高度 \\
    \verb|line1 \\ \vspace{5ex} \\ line2| \\
    line1 \\ \vspace{5ex} \\ line2

    \item 换页(新页包含段首缩进) \\
    \verb|line1 \newpage para2| \\
    line1 \newpage para2
\end{itemize}

 \subsection{\LaTeX 控制符}
 \begin{itemize}
    \item \verb|\#| \qquad \#
    \item \verb|\$| \qquad \$
    \item \verb|\%| \qquad \%
    \item \verb|\{| \qquad \{
    \item \verb|\}| \qquad \}
    \item \verb|\&| \qquad \&
    \item \verb|\~{}| \quad \~{}
    \item \verb|\_{}| \quad \_{}
    \item \verb|\^{}| \quad \^{}
    \item \verb|\textbackslash| \qquad \textbackslash
 \end{itemize}

 \subsection{排版符号}
 \S \P \dag \ddag \copyright \pounds

 \subsection{\TeX 标志符号}
 \LaTeX \TeX{}  \LaTeXe{}

 \subsection{引号}

 \paragraph*{半角输入模式:}

 \begin{itemize}
    \item \verb|`| \ \quad 左单引号:`        %(键盘第二行第一个)  
    \item \verb|'|  \ \quad 右单引号:' 
    \item \verb|``| \quad 左双引号:`` 
    \item \verb|''| \quad 右双引号:'' 
 \end{itemize}

 \paragraph*{全角输入模式:}

\begin{itemize}
    \item  ‘  \ \quad 左单引号:‘ 
    \item  ’  \ \quad 右单引号:’ 
    \item  \verb|“| \ \quad 左双引号:“ 
    \item  \verb|”| \ \quad 右双引号:” 
\end{itemize}

 \subsection{连字符}

\begin{itemize}
    \item \verb|-| \ \ \ \quad -
    \item \verb|--| \ \ \quad --
    \item \verb|---| \quad ---
\end{itemize}

\section{各节一级标题 section}

\textbf{注意:命令不带*会显示编号,带*则不进行编号}

\subsection{各节二级标题 subsection}
你是内容

\subsection*{无编号二级标题 subsection*}
content

\subsubsection{各节三级标题 subsubsection}
他是内容

\paragraph{四级标题 paragraph}
内容内容

\subparagraph{五级标题 subparagraph}
内容内容

\section{字体样式}

\subsection{字体调节}

\begin{tabular}{ll}
	\verb|\songti|   & {\songti 宋体}   \\
	\verb|\heiti|    & {\heiti 黑体}    \\
	\verb|\fangsong| & {\fangsong 仿宋} \\
	\verb|\kaishu|   & {\kaishu 楷书}
\end{tabular}


\subsection{字号调节}
字号命令: \verb|\zihao| \index{zihao}

\begin{tabular}{ll}
\verb|\zihao{0}| &\zihao{0}  初号字 English \\
\verb|\zihao{-0}|&\zihao{-0} 小初号 English \\
\verb|\zihao{1} |&\zihao{1}  一号字 English \\
\verb|\zihao{-1}|&\zihao{-1} 小一号 English \\
\verb|\zihao{2} |&\zihao{2}  二号字 English \\
\verb|\zihao{-2}|&\zihao{-2} 小二号 English \\
\verb|\zihao{3} |&\zihao{3}  三号字 English \\
\verb|\zihao{-3}|&\zihao{-3} 小三号 English \\
\verb|\zihao{4} |&\zihao{4}  四号字 English \\
\verb|\zihao{-4}|&\zihao{-4} 小四号 English \\
\verb|\zihao{5} |&\zihao{5}  五号字 English \\
\verb|\zihao{-5}|&\zihao{-5} 小五号 English \\
\verb|\zihao{6} |&\zihao{6}  六号字 English \\
\verb|\zihao{-6}|&\zihao{-6} 小六号 English \\
\verb|\zihao{7} |&\zihao{7}  七号字 English \\
\verb|\zihao{8} |&\zihao{8}  八号字 English \\
\end{tabular}

\subsection{字体样式}

注意:部分字体不支持会出现Warning警告,请根据实际情况进行使用。

宋体\quad \textbf{粗体}\quad \textit{斜体}\quad \textbf{\textit{粗斜体}}。

{\heiti 黑体\quad \textbf{粗体}\quad \textit{斜体}\quad \textbf{\textit{粗斜体}}}。

{\fangsong 仿宋\quad \textbf{粗体}\quad \textit{斜体}\quad \textbf{\textit{粗斜体}}}。

{\kaishu 楷书\quad \textbf{粗体}\quad \textit{斜体}\quad \textbf{\textit{粗斜体}}}。

Serif\quad \textit{Italic}\quad \textbf{Bold}\quad \textbf{\textit{BoldItalic}}

{\sffamily Sans\quad \textit{Italic}\quad \textbf{Bold}\quad \textbf{\textit{BoldItalic}}}

{\ttfamily Mono\quad \textit{Italic}\quad \textbf{Bold}\quad \textbf{\textit{BoldItalic}}}


\section{常用命令}

\begin{description}
  \item[cite]  参考文献引用, 得到形如 [3-7] 的样式。
  \item[color,xcolor]  支持彩色。
  \item[enumerate]  方便自由选择 enumerate 环境的编号方式。\\
  \textbf{详细说明请见\ref{sec:list}节。} \\
  比如

  \verb|\begin{enumerate}[(a)]| 得到形如 (a), (b), (c) 的编号。


  \verb|\begin{enumerate}[i)]| 得到形如 i), ii), iii) 的编号。

  \verb|\begin{enumerate}[\hspace{1cm}(1)]| \verb|\hspace|命令用于调整距离。

\end{description}

另外要说明的是, itemize, enumerate, description 这三种 list 环境,已经调节了其间距和缩进,
以符合中文书写的习惯。


\section{引用}

\textbf{详细说明请见\ref{cha:ref}节。}

参考文献的引用, 用命令~\verb|\cite{ }|。大括号内要填入的字串, 是自命名的文献条目名。参考文献建议使用文献管理工具进行管理,将所需文献导出bib格式,放于ref目录下,并在main.tex中引入。论文最后的参考文献会根据bib文件与实际cite的项目自动生成。

比如, 通常我们会说:

 {\kaishu
关于此问题, 请参见文献 \cite{oclc2000about}. 作者某某还提到了某某概念\upcite{xiaoyu2001chubanye}.}


上文使用的源文件为:

 {\kaishu
关于此问题, 请参见文献~\verb|\cite{oclc2000about}|. 作者某某还提到了某某概念~\verb|\upcite{xiaoyu2001chubanye}|.
}

其中~\verb|\upcite| 是自定义命令, 使文献引用呈现为\CJKunderdot{上标形式}.

({\heiti 注意:} {\kaishu 这里文献的引用, 有时需要以上标形式出现, 有时需要作为正文文字出现, 为什么?})

另外, 要得到形如~\cite{r1,r3,r4,r5} 的参考文献连续引用, 需要用到 cite 宏包(模板已经加入),
在正文中使用~\verb|\cite{r1,r3,r4,r5}| 的引用形式即可.
或者, 连续引用的上标形式: 使用~\verb|\upcite{r1,r2,r3}|, 得到\upcite{r1,r2,r3}。

若使用VSCode,修改了参考文献后一定要重新编译bib后再使用xe编译,否则cite部分会出现?,找不到引用的参考文献,同时最后参考文献部分也不会显示该引用。

\section{公式}

\textbf{详细说明请见\ref{sec:equation}节。}

公式书写可使用在线编辑器选择合适的命令:\url{https://www.latexlive.com/}

行内公式可使用\verb|$x=x^2$|,即$x=x^2$。

行间公式可使用equation环境,此时一般不建议空一行,会让公式产生缩进不居中。
\begin{equation}
x = 1.
\end{equation}

\section{图形与表格}

\subsection{图形}

\textbf{详细说明请见\ref{sec:figure}节。}

支持对~eps, pdf, jpg 等等常见图形格式。

再次\colorbox{red!45}{澄清一个误会}: \LaTeX{} 支持的图形格式绝非 eps 这一种。 无需特意把图片转化为 eps。

写实验报告时,像素图可直接使用jpg、png等常见格式,矢量图建议转换为pdf并进行适当裁剪后再插入。攥写学术论文时请根据具体刊物的要求插入相应图片。

用形如~\verb|\includegraphics[width=12cm]{Daisy.jpg}| 的命令可以纳入图片。

如图~\ref{fig:1} 是一个纳入~jpg 图片的例子,\verb|[htb]|控制其位置,需要强制将图片放于代码所代表的位置时可以使用\verb|[H]|。

\begin{figure}[H]
\centering
  \includegraphics[width=\textwidth]{Daisy.jpg}
  \caption{一个彩色 jpg 图片的例子}
  \label{fig:1}
\end{figure}

\subsection{表格}

\textbf{详细说明请见\ref{sec:table}节。}

表格问题, 建议使用“三线表”, 如表 \ref{tab:1}。

\begin{table}[htb]
\centering
\caption{一般三线表}
\label{tab:1}
    \begin{tabular}{c c c c c c c c c c c}
    \hline
    123 & 4  & 5  & 123 & 4 & 5123 & 4 & 5 & 123 & 4 & 5\\
    \hline
    67 & 890 & 13 & 123 & 4 & 5123 & 4 & 5 & 123 & 4 & 5\\
    67 & 890 & 13 & 123 & 4 & 5123 & 4 & 5 & 123 & 4 & 5\\
    67 & 890 & 13 & 123 & 4 & 5123 & 4 & 5 & 123 & 4 & 5\\
    \hline
    \end{tabular}
\end{table}



%此处结束正文-------------------------------------------------------------------------------------------------


\include{pages/backmatter} %%%结论

%%%============================================================================================================%%%

%%%=== 参考文献 ========%%%
\cleardoublepage\phantomsection
\addcontentsline{toc}{chapter}{参考文献}
\renewcommand{\baselinestretch}{1.6}
\begin{thebibliography}{00}

  \bibitem{mapreduce} Dean J, Ghemawat S. MapReduce: Simplified Data Processing on Large Clusters[A].Eric A. Brewer, Peter Chen.6th Symposium on Operating Systems Design and Implementation(OSDI 2004)[C], San Francisco, California, USA: {USENIX} Association, 2004:137--150.

  \bibitem{r1} 作者. 文章题目 [J].  期刊名, 出版年份,卷号(期数): 起止页码.

  \bibitem{r2} 作者. 书名 [M]. 版次. 出版地:出版单位,出版年份:起止页码.

  \bibitem{r3} 邓建松等, 《\LaTeXe~科技排版指南》, 科学出版社.

  \bibitem{r4} 吴凌云, 《CTeX~FAQ (常见问题集)》, \textit{Version~0.4}, June 21, 2004.

  \bibitem{r5} Herbert Vo\ss, Mathmode, \url{http://www.tex.ac.uk/ctan/info/math/voss/mathmode/Mathmode.pdf}.


\end{thebibliography}



%%%-------------- 附录. 不需要可以删除.-----------


\appendix

\chapter{测试}

\section{第一个测试}
测试公式编号
\begin{equation}
1+1=2.
\end{equation}

表格编号测试

\begin{table}[h]
  \centering
  \caption{测试表格}
  \begin{tabular}{*{20}c}
     \hline
     % after \\: \hline or \cline{col1-col2} \cline{col3-col4} ...
     11 & 13  & 13  & 13  & 13 \\
     12 & 14  & 13  & 13  & 13 \\
     \hline
   \end{tabular}
\end{table}


\chapter{附录测试}

%%%-------------- 教师评语评分 ------------------
\begin{teacher}
\thispagestyle{empty}
评语: 
\par
\vspace*{12.5cm}
\hspace*{7.5cm}评分: 
\vspace*{1cm}

\hspace*{7.3cm}评阅人:

\vspace*{0.5cm}

\hspace*{10.1cm}年\hspace*{1cm}月\hspace*{1cm}日

\vspace*{0.5cm}

{\songti \zihao{4} \makebox[1cm][s]{(备注:对该实验报告给予优点和不足的评价,并给出百分制评分。)}}

\end{teacher}


\cleardoublepage
\end{document}





